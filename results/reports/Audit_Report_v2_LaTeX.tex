\documentclass[11pt,a4paper]{article}

% Packages
\usepackage[margin=2.5cm]{geometry}
\usepackage{amsmath,amssymb}
\usepackage{graphicx}
\usepackage{booktabs}
\usepackage{float}
\usepackage{hyperref}
\usepackage{cleveref}
\usepackage[utf8]{inputenc}

% Hyperlink settings
\hypersetup{
    colorlinks=true,
    linkcolor=blue,
    urlcolor=blue,
    citecolor=blue
}

% Title information
\title{\textbf{Statistical Audit Report:} \\
Performance Evaluation of CIEDE2000 Across 32 Datasets}
\author{Prepared for CIE Technical Committee Review}
\date{January 2026}

\begin{document}

\maketitle
\tableofcontents
\clearpage
\listoffigures
\listoftables
\clearpage

% ============================================================
\section{Introduction}

\subsection{Objective}

The present study aims to validate the Meta-Color data infrastructure by evaluating the performance of the CIEDE2000 colour-difference formula across 32 experimental datasets spanning a wide range of colour-difference magnitudes and viewing conditions.

\subsection{Methodology}

\subsubsection{Scaling Factor Computation}

For each dataset, a \textbf{Scaling Factor} ($F$) was computed to minimize the least-squares error between the computed colour difference ($\Delta E_{00}$, predicted by CIEDE2000) and the visual colour difference ($\Delta V$, obtained from psychophysical experiments). The scaling factor was determined using Eq.~\ref{eq:scaling}:

\begin{equation}
F = \frac{\sum_{i=1}^{N} \Delta E_{00,i} \cdot \Delta V_i}{\sum_{i=1}^{N} \Delta E_{00,i}^2}
\label{eq:scaling}
\end{equation}

where $N$ represents the number of sample pairs in the dataset.

\subsubsection{Performance Metric}

The analysis focuses on the \textbf{Scaled Ratio}, defined as:

\begin{equation}
R_i = \frac{\Delta E_{00,i}}{F \cdot \Delta V_i}
\label{eq:ratio}
\end{equation}

Under ideal conditions where CIEDE2000 perfectly predicts visual results, $R_i$ should equal 1.0 for all sample pairs.

% ============================================================
\section{Performance Consistency Across Magnitudes}

\subsection{Global Performance Statistics}

The analysis encompassed \textbf{18,137 colour-difference pairs} across 32 datasets. The global statistics are summarized in Table~\ref{tab:global-stats}.

\begin{table}[H]
\centering
\caption{Global performance statistics for the Scaled Ratio distribution}
\label{tab:global-stats}
\begin{tabular}{@{}lr@{}}
\toprule
\textbf{Statistic} & \textbf{Value} \\
\midrule
Global Mean Ratio & 0.9828 \\
Global Standard Deviation & 0.6664 \\
$\pm 1\sigma$ Range & [0.3164, 1.6491] \\
Global Outlier Rate & 1.67\% \\
\bottomrule
\end{tabular}
\end{table}

\subsection{Visual Analysis: Magnitude Dependency}

Figure~\ref{fig:ratio-trend} presents the Scaled Ratio plotted against the computed colour difference ($\Delta E_{00}$), commonly referred to as the ``Ronnier Plot'' in the colour science literature.

\begin{figure}[H]
\centering
\includegraphics[width=0.85\textwidth]{../classic_audit/fig_ronnier_ratio_trend.png}
\caption{Scaled Ratio vs.\ Computed Colour Difference. The red solid line indicates the global mean ratio (0.9828). Blue dashed lines represent $\pm 1$ standard deviation bands (0.3164 to 1.6491). Green dotted lines indicate $\pm 2$ standard deviation bounds. The flat distribution confirms consistent CIEDE2000 performance across magnitude ranges.}
\label{fig:ratio-trend}
\end{figure}

\subsubsection{Interpretation}

The scatter plot reveals \textbf{no significant dependency} of the Scaled Ratio on colour-difference magnitude. Specifically:

\begin{enumerate}
\item \textbf{Absence of Systematic Bias:} The mean line remains approximately horizontal across the magnitude range.
\item \textbf{Homoscedasticity:} The vertical spread appears roughly constant, suggesting uniform prediction error.
\item \textbf{Standard Deviation Bands:} The $\pm 1\sigma$ bands encompass the majority of data points across all magnitudes.
\end{enumerate}

% ============================================================
\section{Dataset Heterogeneity}

\subsection{Outlier Analysis}

Figure~\ref{fig:outlier-ranking} presents the outlier ranking, sorted in descending order of outlier rate.

\begin{figure}[H]
\centering
\includegraphics[width=0.85\textwidth]{../classic_audit/fig2_outlier_ranking.png}
\caption{Outlier Rate by Dataset. Datasets are ranked by the percentage of sample pairs with Scaled Ratios outside the $\pm 1\sigma$ range. The WCG dataset exhibits the highest outlier rate (25\%), followed by Parametric-NS (10\%) and BIGC-T2-SG (5\%). The remaining 29 datasets show outlier rates below 3\%, consistent with normal observer variability.}
\label{fig:outlier-ranking}
\end{figure}

\subsection{High-Noise Datasets}

The top three datasets with the highest outlier rates are summarized in Table~\ref{tab:outliers}.

\begin{table}[H]
\centering
\caption{High-noise datasets and hypothesized causes of elevated outlier rates}
\label{tab:outliers}
\begin{tabular}{@{}llp{6cm}@{}}
\toprule
\textbf{Dataset} & \textbf{Outlier Rate} & \textbf{Hypothesized Cause} \\
\midrule
WCG & 25.0\% & Extreme chromaticity at spectral locus, high saturation \\
Parametric-NS & 10.2\% & Simultaneous contrast effects (no separation paradigm) \\
BIGC-T2-SG & 5.1\% & Gloss-related effects, specular reflections \\
\bottomrule
\end{tabular}
\end{table}

% ============================================================
\section{Conclusion}

\subsection{Summary}

The present audit successfully harmonized 32 colour-difference datasets using dataset-specific scaling factors. Key findings include:

\begin{itemize}
\item \textbf{Scaling Factor Robustness:} The global mean ratio of 0.9828 indicates accurate normalization.
\item \textbf{Magnitude Independence:} No systematic dependency observed across colour-difference magnitudes.
\item \textbf{Data Integrity:} Confirmed. Outliers reflect genuine CIEDE2000 limitations rather than data errors.
\end{itemize}

\subsection{Recommendations for Module 2}

\begin{enumerate}
\item Implement dataset weighting using inverse-variance weighting.
\item Investigate parametric effects (separation, gloss, gamut).
\item Benchmark alternative formulae (CAM16-UCS) on high-noise datasets.
\end{enumerate}

% ============================================================
\section*{References}
\addcontentsline{toc}{section}{References}

\begin{itemize}
\item CIE 217:2016. \textit{Recommended Method for Evaluating the Performance of Colour-Difference Formulae}. Vienna: CIE Central Bureau.
\item Luo, M.~R., Xu, Q., Pointer, M., Melgosa, M., Cui, G., Li, C., Xiao, K., \& Huang, M. (2023). A comprehensive test of colour-difference formulae and uniform colour spaces using available visual datasets. \textit{Color Research \& Application}, 48(3), 267--282.
\item Wang, H., Cui, G., Luo, M.~R., \& Xu, H. (2012). Evaluation of colour-difference formulae for different colour-difference magnitudes. \textit{Color Research \& Application}, 37(5), 316--325.
\end{itemize}

\end{document}
